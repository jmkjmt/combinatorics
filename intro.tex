\section{Introduction}
\label{sec:preliminaries}
프로그램의 termination을 판단하는 문제는 halting problem과 관련이 있다.
Halting problem은 주어진 프로그램이 주어진 입력에 대해 종료하는지 판단하는 문제로, Alan Turing이 1936년에 제안한 문제이다~\cite{turing}.

이 문제는 다음과 같다.
\begin{theorem}
    there can be no machine which, supplied with any one
F of these formulae, will eventually say whether F is provable.
\end{theorem}

그러나 Halting problem이 시사하는 바는 모든 프로그램에 대해 termination을 판단하는 algorithm이 존재하지 않는다는 것이지, 모든 프로그램의 termination을 보일 수 없다는 것이 아니다.
따라서 지금까지 최대한 많은 프로그램에 대해서 termination을 증명하고자 하는 시도가 있었다.
그리고 이런 노력은 order theory와 관련이 있다.
해당 report는 이러한 부분에 주목하여 order theory가 program termination 증명에서 어떤 역할을 하는지 알아보고자 한다.