\section{Introduction}
\label{sec:intro}
The relationship between program termination and the Halting Problem is a central question in the theory of computation.
The Halting Problem, introduced by Alan Turing~\cite{turing}, asks whether it is possible to determine, for any given program and input, whether the program will eventually terminate or run forever.

A partial extract from Turing's original work is as follows:
\begin{theorem}
    There can be no machine which, supplied with any one
$\mathfrak{A}$ of these formulae, will eventually say whether $\mathfrak{A}$ is provable.
\end{theorem}
To put it simply, this can be restated as:
\begin{theorem}
    No algorithm exists that determine whether terminate or not for all possible programs.
\end{theorem}

While the Halting Problem establishes that no algorithm can universally decide the termination of all programs, it does not rule out the possibility of proving termination for specific programs.
As such, significant research has focused on developing methods to establish termination proofs for as many programs as possible.
These efforts are closely linked to the principles of order theory.
This report examines how order theory plays a crucial role in proving program termination, with a particular emphasis on its applications in formal verification and termination analysis.