\section{Introduction}
\label{sec:preliminaries}
Determine whether program terminate is relate to haltin problem.
Halting problem is 
Halting problem is a question of determining whether a given program ends for a given input, proposed by Alan Turing~\cite{turing}.

A partial extract from \cite{turing} is as follows:
\begin{theorem}
    There can be no machine which, supplied with any one
$\mathfrak{A}$ of these formulae, will eventually say whether $\mathfrak{A}$ is provable.
\end{theorem}

To make this easier to understand, it is as follows.
\begin{theorem}
    No algorithm exists that determine whether terminate or not for all possible programs.
\end{theorem}

However, the Halting Problem implies that no algorithm can universally determine the termination of all programs, but it does not preclude the possibility of proving the termination of specific programs.
Consequently, significant efforts have been made to establish termination proofs for as many programs as possible.
These endeavors are closely related to order theory.
This report focuses on examining the role of order theory in proving program termination.