\section{Several Topics in Program Termination}
이 섹션에서는 프로그램의 termination을 증명하는데 사용되는 여러 방법들을 소개한다.
\label{sec:topics}
\subsection{Well-founded Relation}
이 장에서는 termination을 formal하게 다루는데 사용되는 well-founded relation에 대해 설명한다.
다음 정의를 보자:~\cite{order}

\begin{definition}
    A partially ordered set $(\well, \leq)$ is called well-founded, 
    if $\well$ has no infinite descending chain $\{a_0, a_1, \dots, a_n, \dots \mid n < \omega\}$, i. e. with
    $a_0 > a_1 >a_2 > a_3 > \dots$ \newline
    And, $<$ is called a well-founded relation.
\end{definition}
\begin{example}
    자연수 집합 N 에 대해 부등호 $<$ 를 부여하면 (N,<)는 well-founded set이다.
\end{example}

\subsection{Program Termination}
A program π, possibly nondeterministic, is terminating if every computation of π
from an initial state is finite. If there is a computation of π from state x to state
y, we say that y is reachable from x and write y ≺ x. A state y is reachable if it
is reachable from an initial state ~\cite{term}


그러나 program의 state S 와 well-founded relation R 을 정의하는 것은 어려운 작업이다.
따라서 Ranking Function을 도입한다.
ranking function은 S -> N 으로 매핑되는 함수로, termination을 증명하는데 사용된다.
이는 program state S 를 well-founded set인 자연수 N으로 매핑하는 함수로서
well-founded relation을 명시적으로 구하는 것이다.

\begin{example}
    sigma
\end{example}
