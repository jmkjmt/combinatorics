\section{Several Topics in Program Termination}
This section introduces major methods used to prove program termination.
\label{sec:topics}
\subsection{Well-founded Relation}
First, let's consider following definition:~\cite{order}

\begin{definition}
    A partially ordered set $(\well, \leq)$ is called well-founded, 
    if $\well$ has no infinite descending chain $\{a_0, a_1, \dots, a_n, \dots \mid n < \omega\}$, i. e. with
    $a_0 > a_1 >a_2 > a_3 > \dots\ $ Also, $<$ is called a well-founded relation.
\end{definition}
\begin{example}
    The relation $<$ is well-founded over the $\mathbb{N}$, because any sequnce of natural numbers decreasing according to $<$ is finite: e.g.,\\
    $999 > 400 > 388 > ... > 3 > 1 > 0$
\end{example}
\begin{example}
    However, the relation $<$ is not well-founded over the $\mathbb{Q}$ or $\mathbb{R}$.
\end{example}
\subsection{Program Termination}
There is several definition related to program termination form ~\cite{term}.
\begin{definition}
    A program \textbf{P}, possibly nondeterministic, is terminating if every computation of \textbf{P} from an initial state is finite.
\end{definition}
If we can define state of program such as memory state or variable state, following definition is natural.
\begin{definition}
    If there is a computation of \textbf{P} from state x to state y, we say that y is reachable from x and write y $\prec$ x.
    Also, a state y is reachable if it is reachable from an initial state
\end{definition}
\begin{theorem}
    For a given program \textbf{P},
    $\prec$ is well-founded if and only if given program \textbf{P} terminates.
\end{theorem}

However, usually formally defining state of program and well-founded relation $\prec$ is difficult.
So we introduce ranking function to resolve this problem.
\begin{definition}
    ranking function $\delta$ is function that map program state to naturaul number such that
    x $\prec$ y if $\delta$(x) $<$ $\delta$(y) 
\end{definition}
So, if we can find ranking function, we can easily prove program termination.
\begin{example}
    \begin{algorithm}[t]
        \caption{Fibonacci}
        \label{alg:fibonacci}
        \KwIn{natural number n}
        \KwOut{nth element of fibonacci sequence}
        \If{n $<$ 0}{
            \Return 0
        }
        \Else{
            \If{n $<$ 1}{
                \Return 1
            }
            \Else{
                $f1 \gets$ Fibonacci$(n-1)$ \\
                $f2 \gets$ Fibonacci$(n-2)$ \\
                \Return $f1 + f2$
            }
        }
    \end{algorithm}
    Consider Algorithm\ref{alg:fibonacci} that return nth element of fiboncci sequnce.
    In this case, we can define ranking function as just input n.
    At line 7 and 8, Fibonacci function is called recursively.
    So Fibonacci$(n-1)$ and Fibonacci$(n-2)$ is reachable from Fibonacci$(n)$.
    And their ranking function is $n-1, n-2$ repectively.
    Clearly, ranking function is strictly decreasing.
    Therefore, program Fibonacci terminates. 
\end{example}
