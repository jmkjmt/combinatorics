\section{Structural Induction}
\label{sec:structural}
Structural induction is a proof technique used to prove properties of recursively defined objects.
It generalizes mathematical induction to apply to objects that have a recursive structure, such as inductively defined data types or recursively defined functions.
The proof proceeds by induction on the structure of the object.
The base case proves the property for the simplest objects, while the induction step demonstrates that if the property holds for simpler objects, it also holds for more complex objects built from them. The proof is complete when all possible objects are covered by both the base case and the induction step.
Structural induction is widely used in various areas of computer science, including programming language semantics, type theory, and formal verification.
It is especially effective for reasoning about recursive data structures and functions, providing a foundation for rigorous proofs in these domains.
\begin{example}
    In computer science, binary tree can be defined as following language. \\
    \begin{tabular}{rcl}        
    $Tree ::= Empty \mid\ Node\ (value\ Tree\ Tree) $ 
    \end{tabular}
    
    If we want to prove some property of all binary tree, we first prove Empty case, and then, we prove Node (value T1 T2) case with assuming
    T1 and T2 satisfy desired property.
\end{example}